%
%  Origin:  Lehigh CSE Department
%  Course:  CSE 375/475
%  Date:    2013-09-01
%  Version: 1
%
%
%  Description: a basic LaTeX document.  You should use the accompanying
%  makefile to produce a pdf from this source file.  This file is formatted
%  such that as long as the Bibliography is not on the first page, students
%  can be sure that they have produced a sufficient volume of text.  Note
%  that volume is not the basis for a grade, merely a heuristic for ensuring
%  that the writing is not overly cursory
%

%
% Configure a basic LaTeX document with 10pt font and reasonable margins
%
\documentclass[10pt, letterpaper]{article}
\usepackage{fullpage}
\pdfpagewidth=8.5in
\pdfpageheight=11in

%
% Load some useful packages... note that there are many, many, many
% packages... this is just the tip of the iceberg.
%
\usepackage{verbatim, cite, paralist}

%
% Set up a title
%
\title{Critique for Capriccio: Scalable Threads for Internet Services}
\author{Xin Wang}
\date{} % (empty date)

% start the document, make the title
\begin{document}
\maketitle

\section{Summary}
The paper "Capriccio: Scalable Threads for Internet Services"\cite{von2003capriccio} presents a
scalable thread package on high-concurrency servers in order to satisfy the increasing scalability demands.
Current comodity hardware has enabled servers to handle tons of simultaneous connections while current software
can hardly
achieve the same goal without significant performance degradation. Though event-based systems can handle
requests using a pipeline of stages, such systems have drawbacks such as hiding control flow which mades
examining source code and debugging difficult compared to thread-based systems. The contribution of this paper
is to fix the thread-based model to achieve high concurrency performance instead of using event-based one.
The goals include supporting existing thread APIs, enable large scalability of threads and flexibility to
address application-specific needs. The new thread package Capriccio proposed in this paper helps to improve the
scalability of basic thread operations, to solve stack allocation problem for large number of threads by
introducing linked stacks and to extract control flow information to make scheduling decisions by a
resource-aware scheduler.

Capriccio is a user-level thread package which supports POSIX API. The reason why it chooses user-level threads
is that user-level threads can provide flexibility and good performance when such threads greatly reduce the
overhead of thread synchronization, avoid requiring kernel crossings for mutex acquisition or release and make
memory management more efficient. On the other hand, user-level threads are disadavantagable when retaining
control of the processor during a blocking I/O call execution, when introducing a wrapper layer that translates
blocking I/O mechanisms to non-blocking I/O ones, and when taking advantage of multiple processors.

Capriccio is implemented on top of Edgar Toernig's coroutine library that provides fast context switches.
Capriccio also intercepts blocking I/O calls at the library level, which works flawlessly for both dynamically
and statically linked applications. Capriccio provides a modular schedularing mechanism that hides the
event-driven behavior from the programmer and allows users to easily select between different schedulers.
Capriccio takes advantage of cooperative scheduling in order to improve synchronization. 

As to linked stack management, compiler analysis is used to limit preallocated stack space amount. The analysis
is based on a weighted call graph, with which a reasonable bound on the amount of stack space consumed by the
threads. To implement dynamically-sized stacks, the call graph analysis identifies call sites where checkpoints
are inserted. When placing checkpoints, we need a bound on the stack space that should be consumed before our
reaching the next checkpoint. Since wasted stack space may happen when a new stack chunk is linked or exist at
the bottom of current chunk, users are allowed to tune the algorithm for fewer wated space and faster execution
speed. The linked stack algorithm has made preallocation of large stacks unnecessary therefore reduces vitual
memory presure; this algorithm also improves paging behavior and reduces overal memory waste.

For resource-aware scheduling, Capriccio provides application-specific scheduling for thread-based applications
so that we can use the automated scheduling to control admission and to improve response time. The key
abstraction for scheduling is blocking graph. Capriccio generates this graph by observing the transitions between
blocking points. Caprissio can learn the application dynamically and improve scheduling and admission control
accordingly. Capriccio also introduces resource-aware scheduling to keep track of resource utilization levles and
decide if resource reaches the limit, annotate each node with resources to predict the resource's impact, and to
prioritize nodes for scheduling.

\section{Positive reactions}
First, Capriccio has the advantages of user-level threads such as flexibility, being lightweighted, inexpensive
synchronization and fast context switches. Second, the most significant advantages of Capriccio are its scalability, efficient stack management and resource-aware
scheduling. According to the experimental results, the fixing thread package of Capriccio is a good solution to
the problem of establishing scalable and high-concurrency servers. Resource-aware scheduling and linked stacks
also successfully make performance improvements and help to achieve scalability compared to event-based and
thread-based systems. 

\section{Negative reactions}
First, Capriccio system ensures that threads be made as memory efficient as events-based system. This requires
modification to C compiler so that the compiler can perform dynamic memory allocation of stack space. However,
this is not quite convenient since dynamic memory allocation may rapidly cause memory fragmentation in
memory-constrained devices so that the dynamic allocation of stack memory is not very feasible. 

Second, the Capriccio thread package implements resource-aware scheduling by monitoring behavior of the threads.
The thread package also measures the resource requirements of the threads. Then scheduling decisions are made
accordingly and resource utilization is optimized. Such method is a bit generalized than multithreaded chip
multiprocessors as mentioned in\cite{seltzer2005performance}, since it can optimize for usage of different types of resources, yet
requires detailed monitoring of the program state.

\section{Extension proposal}
Based on the implementation of Capriccio, future programmers can provide more information related to high-level
structure of the servers' tasks to compiler. The prorgrams may be written in threaded style.
\cite{von2003capriccio}

\section{Conclusions}
To sum up, Capriccio is a good proposal for high-concurrency servers and has helped to deal with many empirical
problems such as scalability and scheduling. It also provides us with good insight about how new techniques can
be integrated to compiler.

\bibliographystyle{plain}
\bibliography{cse375}
\end{document}
