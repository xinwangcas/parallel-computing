%
%  Origin:  Lehigh CSE Department
%  Course:  CSE 375/475
%  Date:    2013-09-01
%  Version: 1
%
%
%  Description: a basic LaTeX document.  You should use the accompanying
%  makefile to produce a pdf from this source file.  This file is formatted
%  such that as long as the Bibliography is not on the first page, students
%  can be sure that they have produced a sufficient volume of text.  Note
%  that volume is not the basis for a grade, merely a heuristic for ensuring
%  that the writing is not overly cursory
%

%
% Configure a basic LaTeX document with 10pt font and reasonable margins
%
\documentclass[10pt, letterpaper]{article}
\usepackage{fullpage}
\pdfpagewidth=8.5in
\pdfpageheight=11in

%
% Load some useful packages... note that there are many, many, many
% packages... this is just the tip of the iceberg.
%
\usepackage{verbatim, cite, paralist}

%
% Set up a title
%
\title{Critique for Hoard: A Scalable Memory Allocator for Multithreaded Applications}
\author{Xin Wang}
\date{} % (empty date)

% start the document, make the title
\begin{document}
\maketitle

\section{Summary}
Many memory allocators have been bottlenecks for parallel, multithreaded C and C++ applications in limiting their program
scalability and performance. Problems with memory allocators include low speed and scalability, false
sharing caused by heap organizations, and increase in memory consumption when confronting producer-consumer
object allocation. The paper Hoard: A Scalable Memory Allocator for Multithreaded Applications has proposed a
better allocator with rapid speed and good scalability, that can avoid false sharing and is memory
efficient. Hoard is noticable superior to other techniques in solving blowup and false sharing problems, and
achieves barely any synchronization costs.\cite{saha2006mcrt} 

Hoard is designed to overcome the problems with previous allocators such as false sharing that occurs in heap
objects when cache lines are divided into small objects. It also keeps blowup to a bounded constant factor, while
other allocators are unable to bound blowups. First, in order to bound blowup, Hoard maintains a global heap as well
as heaps for each processor. It transfers a large chunk of memory from per-processor heap to global heap, which
can be reused by other processors. Each thread can not access global heap and per-processor heaps at the same
time. Hoard maintains usage statistics for each heap and allocates memory in superblocks. All blocks in a
superblock are in the same size class. Whenever a per-processor heap reaches the emptiness threshold, Hoard moves
superblocks from that heap to the global heap. To improve locality, when a block in a superblock is freed, the
superblock is moved to the front of its fullness group. Second, Hoard combines superblocks together with
multiple-heaps to avoid false sharing. Hoard ensures that only one thread can allocate from a given superblock. If
more than one thread make requests from memory, these requests should be satisfied from different superblocks.
Once a block is deallocated, Hoard returns the block to its superblock. 

A set of experiments have be designed to prove the effectiveness of Hoard. These experiments demonstrate the fast
speed, good scalability, false sharing avoidance ability and low fragmentation of Hoard and make comparison
between Hoard and other memory
allocators such as Solaris, Ptmalloc, MTmalloc etc. The experiments are carried both on
uniprocessors and multiprocessors. The results are stable and robust. 

To sum up, this paper has introduced a new memory allocator Hoard, which is able to improve performance such as
speed, scalability, false sharing avoidance and low fragmentation, compared to previous allocators. Its
per-processor and global heaps as well as moving superblocks across heaps have enabled these features and help to
bound blowup.
\section{Positive reactions}
Hoard is designed to overcome the disadvantages of previous memory allocators. It is the first allocator that has significantly avoided the
false sharing and memory inefficient problems. Hoard uses novel techniques of organizing per-processor heaps and
global heap and moving superblocks among heaps. These techniques have helped Hoard to perform better than others
not only in the aspects mentioned above, but also in bounding blowup and reach low expected case synchronization.

The experiments in the paper are convincing and the results are successful. The improvement in experiments in speed,
scalability, false sharing avoidance and low fragmentation has proved Hoard a better memory allocator than other
allocators, which are all useful and prevalent tools, involved in the experiments. Robustness has also be proved
from the results. The sensitivity study helps to examine the effect of altering empty fraction on runtime and
fragmentation for multithreaded benchmarks.

According to the current requiring for scalable architecture and runtime system support, Hoard has provided
researchers with a good direction to this field.

\section{Negative reactions}
Hoard is designed to deal with problems of memory allocators for parallel and multithreaded C and C++ programs.
In C and C++ programs, when allocating, deallocating or reallocating memory, pointers need to be stored
appropriately. However, Hoard does not take copying collectors and pointers storage into consideration. 

As indicated in \cite{hudson2006mcrt}, Hoard may rely heavily on CAS(compare and swap) hardware instruction. CAS instructions
can be several orders of magnitude slower than non-atomic operations. Therefore, it is meaningful to try to avoid
atomic operations in common malloc and free routines and to use them in non-blocking way.

\section{Extension proposal}
If we take transaction memory into consideration, Hoard can be extended to support transaction that enable
transactional use of the system memory allocator.

\section{Conclusions}
To sum up, Hoard has contributed much in improving the performance of memory allocation and has successfully
dealt with many problems with other previous allocators, which is proved by experiments in the paper. This paper
has big influence to future research. For example, according to similar ideas, Olszewski et al. have implemented JudoSTM
\cite{olszewski2007judostm}, a novel dynamic binary-rewriting technique to implemnt STM supporting C/C++
programs. 

\bibliographystyle{plain}
\bibliography{cse375}
\end{document}
