%
%  Origin:  Lehigh CSE Department
%  Course:  CSE 375/475
%  Date:    2013-09-01
%  Version: 1
%
%
%  Description: a basic LaTeX document.  You should use the accompanying
%  makefile to produce a pdf from this source file.  This file is formatted
%  such that as long as the Bibliography is not on the first page, students
%  can be sure that they have produced a sufficient volume of text.  Note
%  that volume is not the basis for a grade, merely a heuristic for ensuring
%  that the writing is not overly cursory
%

%
% Configure a basic LaTeX document with 10pt font and reasonable margins
%
\documentclass[10pt, letterpaper]{article}
\usepackage{fullpage}
\pdfpagewidth=8.5in
\pdfpageheight=11in

%
% Load some useful packages... note that there are many, many, many
% packages... this is just the tip of the iceberg.
%
\usepackage{verbatim, cite, paralist}

%
% Set up a title
%
\title{Critique of A Bridging Model for Parallel Computation}
\author{Xin Wang}
\date{} % (empty date)

% start the document, make the title
\begin{document}
\maketitle

\section{Summary}
The paper "A Bridging Model of Parallel Computation"\cite{valiant-cacm-1990}
introduces a bridging model between hardware and software programming, called bulk-synchronous
parallel(BSP) model. The BSP model is a promising bridging model for general-purpose parallel computation and it
aims to enable parallel programs to be executed efficiently on a variety of architectures. The BSP
computers(BSPC) include three components: some unit processors for computation or memory
functions, a router to communicate and deliver messages among the processors, and barrier synchronization among a
set of processors at L time units. The computational tasks and communication can be performed by components
separately. In each superstep, different components work seperately, then send messages and wait for
synchronization before moving to the next superstep.
Then the automatic memory management on BSPC, the concurrent memory accesses on BSPC, BSP algorithms without
hashing, the implementation on packet switching networks and implementation on optical crossbars are also
discussed. Advantages of BSP model lie in three aspects. The first advantage is the good programmability given
balanced computational and communication bandwidth. Second, BSP model provides a good platform for important
algorithms and controls memory management overheads. Third, BSP model can be related to different technology and
topology on networks.
\section{Positive reactions}
\label{sec-formatting}
The BSP model provides an innovative idea for establishing bridging model between hardware and software
programming in parallel computation in 1990. The superstep concept enables computation and communication
actions wait for each other without strict order. The superstep ends with barrier synchronization which ensures all
one-sided communications. In theory, the BSP model enables asymptotically optimal execution on a variety of
computer architectures. As mentioned above, BSP model is advantagable in its programmability, portability and
efficiency. The success of Apache Hama, Apache Giraph and Pregel has demonstrated the merits of BSP
model.
The authors have proposed some possible extensions of BSP model such as parallel-prefix computation BSP, broadcasting
BSP and concurrent reads and writes BSP\cite{azizoglu2000lower}. In fact, a variety of more optimization
algorithms and
architectures can be applied to the platform that BSP provides so that different level of parallism are
implemented. For example, sorting the processors that
are currently working in local memory according to an optimal time sequential algorithm so that the remaining
tasks are balanced and efficiently partitioned. A dynamic programming algorithm can also be applied so that
overload components can share tasks with idle ones. Ref\cite{Gerbessiotis96deterministicsorting} is an example of
this kind of extented BSP model with deterministic optimization algorithms. 
\section{Negative reactions}
Even though the BSP model is carefully designed and simulations for BSP have provided good results, certain limitations
can still be found in the initial version of this model. 
First, BSP model is established according to a batch-routing interconnection network without imbalanced communication problems.
However, in real world, different kinds of underlying network topology should not be ignored. The model should take more
complex topology into consideration rather than network consisting only of local processors and non-local ones.
Efficient rounting algorithms that are designed specifically to different level of network topology may improve
efficiency. For example, when large groups of processing components from different places are clustered to form a more complex
topology, a hierarchical topology can be considered. The hierarchical structure may need different level of
synchronization in subsets and therefore lead to different level of parallelism.
Second, complete communication among all processors may sometimes be less efficient especially when the message
amount is small and routing topology is complex and time consuming. Instead of sending one message after another
one as it may be the case in BSP model, routing large amoung of messages together may further improve efficiency. 

\section{Extension proposal}
Though some useful BSP libraries have been implemented for different purposes and structures already, design and
implement one library with good interface for users are very helpful. Such an interface should free users from worrying
about the network topology, the workload on different processors, and the rounting phase contention. At the same
time, dynamic optimal algorithms are called automatically to fit specific structures and tasks. The
interface should also provides a virtual shared memory image for users.
%
% Note: we can pull in more files!
%

\section{Conclusions}
To sum up, the paper proposes a useful bridging model for parallel computation with good adaptivity to different
platforms, programs and structures as well as with efficiency. It also provides useful ideas to extend the model
for different parallism scheme.
\bibliographystyle{plain}
\bibliography{cse375}
\end{document}
