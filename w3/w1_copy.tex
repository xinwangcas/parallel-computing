%
%  Origin:  Lehigh CSE Department
%  Course:  CSE 375/475
%  Date:    2013-09-01
%  Version: 1
%
%
%  Description: a basic LaTeX document.  You should use the accompanying
%  makefile to produce a pdf from this source file.  This file is formatted
%  such that as long as the Bibliography is not on the first page, students
%  can be sure that they have produced a sufficient volume of text.  Note
%  that volume is not the basis for a grade, merely a heuristic for ensuring
%  that the writing is not overly cursory
%

%
% Configure a basic LaTeX document with 10pt font and reasonable margins
%
\documentclass[10pt, letterpaper]{article}
\usepackage{fullpage}
\pdfpagewidth=8.5in
\pdfpageheight=11in

%
% Load some useful packages... note that there are many, many, many
% packages... this is just the tip of the iceberg.
%
\usepackage{verbatim, cite, paralist}

%
% Set up a title
%
\title{LaTeX Instructions for CSE 375/475}
\author{Michael Spear}
\date{} % (empty date)

% start the document, make the title
\begin{document}
\maketitle

\section{Introduction}
Using a pure text format for writing documents makes it much easier to
version control them.  For that reason, this example document is being
provided to show some basic features of LaTeX.

\section{Formatting Suggestions}
\label{sec-formatting}
Here are some formatting suggestions:

\subsection{Fonts}
\label{sec-fonts}
The philosophy in LaTeX is that you shouldn't mess with formatting unless
absolutely necessary.  While it's fine to put text in \textbf{bold},
\textit{italics}, and \texttt{code format}, in general you shouldn't mess
around with font sizes.\footnote{Note that footnotes are automatically in a
  smaller font.}

\subsection{Verbatim Formatting}
\label{sec-verb}
Similar to the HTML \verb|<pre>| tag, you can use in-line verbatim mode or
the block formatted verbatim mode to format code and other pre-formatted
text.  Here's an example:

\begin{verbatim}
for (auto x : myvector) {
  foo(x);
}
\end{verbatim}

\subsection{Lists}
There are simple list and enumeration modes.  You should never use them,
because they create unnecessary whitespace.  Instead, use the compact
variants as shown below:

Things I dislike:
\begin{compactitem}
\item Lists
\item Lame attempts at irony
\item Using someone else's joke without citation.
\end{compactitem}

Steps to success:
\begin{compactenum}
\item Collect underpants.
\item ???
\item Profit.
\end{compactenum}

\subsection{Citations}
Citations are very powerful.  Once you put an entry in your .bib file, you
can use it anywhere, any time, and LaTeX does all the work.  For example,
note that the first assignment requires us to read a paper by
Valiant~\cite{valiant-cacm-1990}.

\section{Labels and References}
\label{sec-labels}
Every section, subsection, and so forth can be given a label.  This makes it
easy to get internal references right.  For example, if one were to add more
sections and subsections, the document would still get references right:
fonts are discussed in subsection~\ref{sec-fonts}, and verbatim formatting is
discussed in subsection~\ref{sec-verb}.

\section{Equations}
LaTeX can do amazing things with equations.  For example, $\forall_{x \in
  myvector} {ans = \Sigma_{i=1}^{i=72}foo(x,i)}$

\begin{equation}
\label{eqn-asb}
\alpha = \sqrt{ \beta }
\end{equation}

Some people believe that Equation~\ref{eqn-asb} is very useful.  Others
complain that the indentation of all paragraphs within a section, except for
the first paragraph, leads to funny-looking text.  It's best to just ignore
such oddities for the purposes of this class.

%
% Note: we can pull in more files!
%
\section{Files}
It's very easy to split your document into many files.  This makes
collaboration and version control much easier.


\section{Conclusions}
There are many good tutorials online.  This is meant to show just enough to
get students started with LaTeX for CSE 375/475.  Hopefully it will be an
enjoyable tool to learn.

\bibliographystyle{plain}
\bibliography{cse375}
\end{document}
